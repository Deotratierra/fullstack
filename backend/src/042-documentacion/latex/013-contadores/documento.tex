\documentclass{article}

\usepackage[spanish]{babel}
\usepackage[utf8]{inputenc}

\iffalse
LaTeX utiliza un amplio número de contadores con el fin de enumerar
distintos elementos de un documento: páginas, secciones, tablas, figuras...
Cada contador tiene un nombre que permite identificarlo; así, page es
el contador que identifica páginas, chapter capítulos, etc...

Se dispone de los siguientes formatos de contador:
\arabic{NombreContador}     1, 2, 3, 4...
\alph{NombreContador}       a, b, c, d... z    (máximo 27)
\Alph{NombreContador}       A, B, C, D... Z    (máximo 27)
\roman{NombreContador}      I, II, III, IV...  (con babel{spanish}, si no: i, ii, iii...)
\Roman{NombreContador}      I, II, III, IV...
\fnsymbol{NombreContador}   *, **, ***, ****   (con babel{spanish}, si no: *, †, ‡...)

Asociado a cada contador existe un comando, llamado representación del contador,
que permite imprimir el valor del contador NombreContador en alguno de los
formatos descritos; el comando es:
\theNombreContador
\fi

% ===================================================

\begin{document}

% Mostrar el número actual de un contador
Podemos obtener el número de la página en la que estamos: \thepage  % 1

% Redefinir la representación de un contador:
\renewcommand{\thepage}{\roman{page}}
Contador redefinido en números romanos: \thepage  % I

% Cambiar el valor de un contador:
\setcounter{page}{3}
Ahora estamos en la página \thepage  % III

% Incrementar el contador en x cantidad:
\addtocounter{page}{6}
Ahora estamos en la página \thepage  % IX

% ===================================================

\iffalse
% Definir un nuevo contador:
\newcounter{Contador}[Contador existente]

El parámero de contador existente es opcional. Sirve para subordinar Contador
a otro contador existente. Por ejemplo, el contador subsection está subordinado
al contador section de tal forma que si el contador section se incrementa en 1
el contador subsection se reinicia automáticamente.

Al crear Contador se crea el comando \theContador con la definición \arabic{Contador}
por defecto.
\fi

\end{document}

\iffalse
Fuentes:
http://metodos.fam.cie.uva.es/~latex/apuntes/apuntes8.pdf
\fi