\documentclass{article}

\usepackage[utf8]{inputenc}

\begin{document}

%     VARIABLES     %

% Definir variables (nuevos comandos):
\newcommand{\x}{30}

Para insertar la variable: {\x}

% Redefinir comandos ya existentes:
% \renewcommand

% ====================================================

%     FUNCIONES     %

% Con argumentos posicionales
\newcommand{\saludar}[1] {  % {\comando}[nArgs]{ contenido #1 #2 ... #n argumentos}
    ¡Hola #1!
}

\saludar{Álvaro} % ¡Hola Álvaro!

% ----------------------------------------------

% Con argumentos opcionales
\newcommand{\saludarPorDefecto}[2][Álvaro] { % {\comando}[nArgs][argPorDefecto]
    ¡Hola #1, hola #2!
}


\saludarPorDefecto[a todos]{Cristina}  % ¡Hola a todos, hola Cristina!

\saludarPorDefecto{Cristina}           % ¡Hola Álvaro, hola Cristina!

% ====================================================



\end{document}

\iffalse
Fuentes:
https://en.wikibooks.org/wiki/LaTeX/Macros#New_commands
http://metodos.fam.cie.uva.es/~latex/apuntes/apuntes8.pdf
\fi