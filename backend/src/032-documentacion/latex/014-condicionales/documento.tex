\documentclass{article}

\usepackage[utf8]{inputenc}

% Existen diferentes condicionales dependiendo
% del tipo de variables que usamos

\begin{document}

%     CONDICIONALES NUMÉRICOS     %

% Comparar números
% \ifnum n1 operador n2
\newcommand{\cuatrodigit}[1]{
    \ifnum #1<1000 0\fi
    \ifnum #1<100 0\fi
    \ifnum #1<10 0\fi
    #1
}

\cuatrodigit{125} - \cuatrodigit{8}  % 0125 - 0008

% ------------------------------------------------

% Saber si un número es impar o par
%\ifodd impar \else par \fi
\newcommand{\esImpar}[1]{
  El número #1 \ifodd #1 sí \else no \fi es impar.
}

\esImpar{2}  % El número 2 no es impar

\esImpar{3}  % El número 3 sí es impar


% --------------------------------------------------

% Este tutorial no está terminado

\end{document}

\iffalse
Fuentes:
http://metodos.fam.cie.uva.es/~latex/apuntes/apuntes8.pdf
\fi
