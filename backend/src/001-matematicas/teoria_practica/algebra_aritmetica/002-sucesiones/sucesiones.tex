\documentclass{article}
\usepackage[utf8]{inputenc}

\title{Sucesiones Matemáticas}
\author{Álvaro Mondéjar Rubio}
\date{Diciembre de 2017}

\begin{document}

\maketitle

La sucesiones son conjuntos de números uno detrás de otro. Se utilizan, por ejemplo en los sumatorios.
Siguen una regla determinada por una fórmula. Por ejemplo, según esta sucesión: \{3, 5, 7, 9...\}. La regla sería:
\[ x_n = 2n + 1 \]
Ya que: 
\[ x_1 = 2 \cdot 1 + 1 = 3 \]
\[ x_2 = 2 \cdot 2 + 1 = 5 \]
\[ x_3 = 2 \cdot 3 + 1 = 7 \]
\[ ... \]

Gracias a la fórmula, podemos calcular el término de índice que queramos. Para calcular el término número 500:
\[ x_{500} = 2 \cdot 500 + 1 = 1001 \]


\end{document}
