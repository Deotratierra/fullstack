\documentclass{article}

\usepackage[utf8]{inputenc}
\usepackage{graphicx}
\usepackage{titling}

\setlength{\droptitle}{-10em}  % Elevar el título (require el paquete titling)

\title{El sistema hexadecimal}
\author{}
\date{}

\begin{document}

\maketitle

\section{Introducción}
El sistema hexadecimal (abreviado como "hex") es el sistema de numeración posicional en base 16.

Es muy importante en la informática y las ciencias de la computación debido a que las operaciones de la CPU suelen usar como unidad básica de memoria el byte, que representa $2^8$ valores posibles. Deconstruyéndolo podemos extraer que: $2^8 = 2^4 \times 2^4 = 16 \times 16 = 1 \times 16^2 + 0 \times 16^1 + 0 \times 16^0 = 100_{16}$, es decir que el número $100_{16}$ (en base hexadecimal) representa un byte de memoria.

El conjunto de números (0 al 15) usados en el sistema hexadecimal son:
\begin{center}
$\{0,1,2,3,4,5,6,7,8,9,A,B,C,D,E,F\}$
\end{center}

\subsection{Notación}
Para indicar que una representación equivale a un número decimal, puede aparecer precedida de los caracteres "0x" o seguida del caracter "H".

\section{Conversiones}
\subsection{De hexadecimal a decimal}
\includegraphics[width=1\textwidth]{hexadecimal}

\end{document}