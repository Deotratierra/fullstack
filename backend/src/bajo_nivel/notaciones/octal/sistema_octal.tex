\documentclass{article}
\usepackage[utf8]{inputenc}
\usepackage{blindtext}

% ---   Título y autor   ---
\title{El sistema octal}
\author{Álvaro Mondéjar Rubio}
\date{}

\begin{document}

\maketitle

\section{Introducción}
El sistema numérico en base 8 se llama octal y utiliza los dígitos del 0 al 7. Tiene la ventaja de que no requiere utilizar otros símbolos diferentes de los dígitos.

\section{Conversiones}
\subsection{Octal a decimal}
Supongamos que tenemos el número octal 672,49 y queremos pasarlo a notación decimal. Lo que haríamos sería:
\begin{center}
$2*8^0 + 7*8^1 + 6*8^2 + 4*8^-1 + 9*8^-2 = 2 + 56 + 384 + 0.5 + 0.140625 = 442.640625$
\end{center}

Los números en octal se representan con una "q" (para evitar la confusión entre 0 y o de "octal") junto a ellos para diferenciarlos de los decimales ("d"). Por lo que: $672,49q = 442.640625d$

\end{document}
