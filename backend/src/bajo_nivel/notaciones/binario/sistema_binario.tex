\documentclass{article}

\usepackage[utf8]{inputenc}
\usepackage{graphicx}
\usepackage{multicol} % Necesario para múltiples columnas
\usepackage{titling}

\setlength{\droptitle}{-12em}  % Elevar el título (require el paquete titling)

\title{Sistema binario}
\author{}
\date{}

\begin{document}

\maketitle
\thispagestyle{empty}  % Eliminar la numeración de páginas

\section{Introducción}
El sistema binario es el sistema de numeración posicional de base 2, compuesto por los dígitos 0 y 1 los cuales representan los estados "abierto y "cerrado" en puertas lógicas.

\subsection{Bits}
El nombre "bit" es el acrónimo de dígito binario ("Binary digit") y se refiere a un dígito en el sistema binario. Es la unidad mínima de información empleada en informática, en cualquier diospositivo digital o en teoría de la información.

El número de combinaciones posibles equivale a $2^n$ donde n es el número de bits que poseemos, por ejemplo, si tenemos 2 bits podemos tener 4 estados: $\{0,0\} \{0,1\} \{1,0\} \{1,1\}$  ($2^2 = 8$)

\subsection{Notación}
Para escribir números binarios usamos una "B" seguida del número, para distinguirlo de su homólogo decimal, por ejemplo, el número 218 en binario es: $11011010B$.

\section{Conversiones}
\subsection{De decimal a binario}
Convertir cualquier número en binario simplemente implica detectar los exponentes de 2 en él. Por ejemplo tomemos el número 56. Lo único que debemos de hacer es ir dividiendo por 2 e ir apuntando el resto de derecha a izquierda:
\begin{center}
$56/2 = 28 \hspace{1cm} Resto = 0$

$28/2 = 14 \hspace{1cm} Resto = 0$

$14/2 = 7 \hspace{1cm} Resto = 0$

$7/2 = 3 \hspace{1cm} Resto = 1$

$3/2 = 1 \hspace{1cm} Resto = 1$

Nos quedamos con 1 que no es divisible por 2 así que lo añadimos

$56_{10} = 111000_2$
\end{center}

\subsection{Binario a decimal}
Para convertir de binario a decimal realizamos el siguiente procedimiento: empezamos por el dígito de la derecha del número binario y vamos contando, al lado de cada dígito escribimos la potencia de 2 correspondiente al contador si el número es 1 ó un 0 si el número es 0. Luego sumamos todos los números que hemos obtenido. Por ejemplo:

$111000_2$ \hspace{1cm} Vamos de derecha a izquierda:

$0$ \hspace{1cm} $0$ \hspace{1cm} Sería $2^0 = 1$ pero como es $0$ lo dejamos en $0$

$0$ \hspace{1cm} $0$

$0$ \hspace{1cm} $0$

$1$ \hspace{1cm} $2^3=8$

$1$ \hspace{1cm} $2^4=16$

$1$ \hspace{1cm} $2^5=32$

$111000_2 = 32 + 16 + 8 = 56_{10}$


\thispagestyle{empty}  % Eliminar la numeración de páginas

\begin{multicols}{2}
[
\subsection{Relación entre los sistemas hexadecimal y binario}
El número 218 expresado en hexadecimal es mucho más simple y compacto: 0xDA ya que ($13*16^1 + 10*16^0 = 208 + 10 = 218$).
]

\includegraphics[width=0.44\textwidth]{binary_hex}


Si obervamos la tabla podemos comprobar como, tomando 4 bits, podemos representar todas las combinaciones en hexadecimal, al igual que con 3 bits, todas las combinaciones en octal.

Esto significa que cogemos un número en binario, por ejemplo 1101110 y lo dividimos en grupos de 4 empezando por la derecha: $\{0110, 1110\}$, luego miramos la tabla y podemos representarlo en hexadecimal, que sería 0x6E.

\end{multicols}


\end{document}

\iffalse
Fuentes:
http://blog.refu.co/?p=804
https://es.wikipedia.org/wiki/C%C3%B3digo_binario
\fi